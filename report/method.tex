%!TEX root = paper_s2029200_s2004674_s1974718.tex
\subsection*{Simulation Model}
We want to model emotions as described by the OCC-model [REF!]. This model is quite complex but still favoured among computer scientist to simulate artificial emotions. The implementation used in our experiment will be simplified as the complete model can be used to simultaneously emulate more than 25 emotions. In the paper ``The OCC Model Revisited'' by Steunebrink\cite{}, the model is simplified and further explained.  In our experiment we will only use two (maybe more) to simulate simple interaction. \\
The model has multiple parameters and functions that need to be dissected for each problem researchers tackle. This include events, emotion based functions, praiseworthiness, desirability and more. From Bazzan (2004), we can derive multiple of these functions relevant for the iterated prisoners dilemma (IPD), however the remaining are open for interpretation. \\
For the interactions we are interested in we will use only basic emotions such as fear and joy.\\

As an added layer above these emotions, we will make the simulated agents more robust by adding coalitions. In these coalitions, similar agents will work together to maximize their performance. For actions towards agents outside the coalition, the coalition agents will decide what action will be played by voting. This additional method will teach us if coalitions improve the performance of the individual agents, or improve the performance of the coalition as a whole. Or maybe coalitions will be exploitable for outside agents and decrease the performance all together.
\subsection*{Experiment Design}
Using the IPD we model the interactions between agents. Changing the way the emotions work (such as altering thresholds, sensitivity etcetera) will most likely blow up the experiment. The amount of parameters we can change will result in too much data we cannot process right now. Probably we will alter the ratio between the emotions in the initial state. This will give different or similar results, such as an anarchic state where there are an abundance of defectors, or a peaceful state which reaches an equilibrium with a high number of cooperators. Varying the ratio will allow us to discus the necessity of the initial state to reach a desired end state.