%!TEX root = paper_s2029200_s2004674_s1974718.tex
\subsection*{Problem}
Decisions are almost never purely based on rational thought. Emotions are always involved and influence the decisions you make. As an example, think about the different moods you can be in and try to imagine your willingness to help someone move or play a game. Given that you actually like to play monopoly for example, the chance that you want to play it, is higher when you're in a good mood, given that the rest of the parameters stays the same. \\
In this paper, we want to examine the influence of emotions on decision making and social decision making in particular. We use the iterated prisoners dilemma as a tool to model complex social behaviour and see how different settings influence the way in which agents cooperate.

\subsection*{State of the Art}
Bazzan et all have described how cooperation occurs in agents with the ability to learn. They describe three different ways in which learning agents play the iterated prisoners dilemma. The three different methods how agents play the IPD (The agents learn independently, the agents plays the prisoners dilemma in a hierarchy and the agent plays in a coalition) are all worked out and tested. \\
A few years before that, Bazzan also created a framework for simulating agents with Emotions, using the OCC-model in her framework. 

\subsection*{New Idea}
Our new idea is to combine the two studies mentioned above and simulate agents in a grid that play the iterated prisoners dilemma, while giving them emotions and the ability to form coalitions. Coalitions are formed when a certain number of neighbours express the same emotion as the agent in question. The emotion expressed, is the emotion with the highest value that is above a certain threshold.
